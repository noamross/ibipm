\documentclass{amsart}
\usepackage{graphicx}
\def\N{\mathbb N}
\def\P{\mathbb P}
\def\S{\mathcal S}
\def\n{\mathbf n}
\def\s{\mathbf s}
\def\x{\mathbf x}
\def\y{\mathbf y}
\def\o{\%*\%}
\begin{document}
\title[Establishment of size-structured populations]{Establishment and extinction risk of size-structured populations}
\maketitle

\section*{Introduction}


\section*{General models and methods}
We consider an individual based model where the set of all possible individual states is a compact set $X$. Possibilities for this state space include $X=[a,b]$ for a size-structured population where the minimum size of individuals is $a$ and the maximum size is $b$,  $X=\{1,2,\dots, k\}$ for a population with $k$ discrete stages, $X=\{0\}\cup [a,b]$ for a size-structured plant population with seeds represented by state $0$, or $X=\{1,2,\dots,k\} \times [a,b]$ for populations structured by age and size. As we only consider populations with a finite number of individuals, the state of the population at any point of the time is characterized by the individual states  of individuals within the population  and how many individuals are within each of these states $(n_1,n_2,\dots, n_k)$. For example, if there $n_1$ individuals in states $x_1$, $n_2$ individuals in state $x_2$,..., $n_k$ individuals in state $x_k$, then the state of the population is given by $(n_1,n_2,\dots, n_k; x_1,x_2,\dots, x_k)$. The set of all possible population states is 
\[
\S= \{(\n,\x)=(n_1,\dots, n_k, x_1\dots , x_k): k,n_i\in \N, x_i \in X\} \cup \{0\} 
\]
where $\N=\{1,2,3,\dots\}$ denotes the natural numbers and $0$ is the extinction state i.e. no individuals in the population. 

Let  $\s_t=( \n, \x)\in \S$ be the population state at time $t$. The dynamics of $\s_t$ are determined by a set of probabilistic rules that determine the contribution of each individual in the population to the population in next time step $t+1$. Given let $p_t^x(\n,\y)$ be the probability that an individual in state $x$ at time $t$ contributes $n_1$ individuals in state $y_1$, $n_2$ individuals in state $y_2$, etc. to the population in time $t+1$. These ``contributions'' include individuals changing state, having offspring, or dying in which case there is no contribution with probability $p_t^x(0)$. We assume that individual updates are independent of one other. Under these assumptions, $\s_t$ is what is known as a continuous-state branching process in the probability literature.  These branching processes can be viewed as the demographic stochastic counterparts to integral projection models (IPMs). However, unlike the classical IPMs, these generalized branching process provide insights into how demographic stochasticity influences extinction risk and establishment success. Harris developed a general theory for these branching processes. Here, we discuss some of the highlights of that theory and illustrate the theory with an application to assessing extinction risk for an endangered perennial plant, Menzies wallower.   Since seasonality plays an important role in this population, we extend the theory of Harris to fluctuating environments.

Need some assumptions on $p_t^x(\n,\y)$....erg...need to define this more carefully wrt to a measure


To study the dynamics of $\s_t$ one can either perform individual-based simulations using the probabilistic update rules or analytical methods if one is interested in how the moments of the population and the probability of extinction changes in time. To understand these quantities, it is useful to introduce the probability generating maps (PGMs) associated with the probabilistic update rules. These maps take continuous functions $h:\S\to [0,1]$ to continuous functions of the same type 
\[
\Phi^x_t(h) = \sum_{(\n, \y)\in \S} p_t^x(\n,\y) \prod_i h(y_i)^{n_i}
\]
I think these have the nice property that they generate all the moments
\[
D\Phi_t^x (1)h(y) = 
\]

\subsection*{The Model}
Since  Menzies wallflower is a perennial monocarp, we focus on models for semelparous populations. However, in the Appendices, we describe how to develop models for iteroparous populations. Furthermore, as the data for Menzies wallflower is seasonal, we develop the models for periodic environments. Furthermore, as this population has a mixture of a discrete state (seeds) and continuos states (maximal leaf lengths of individual plants) and we want to write down the simplest representation of the dynamics, we assume that there is a positive, finite measure $\mu$ on $X$ for defining the demographic kernels. For instance, in the case of a purely size structured state space $X=[a,b]$, $\mu$ would be the usual Lebesgue measure on $[a,b]$. Alternatively, as we discuss further later one, if $X=\{0\} \cup [a,b]$ where $\{0\}$ corresponds to the seed class, we choose $\mu$ to be a measure that places a weight of one on $\{0\}$  and Lebesgue measure on $[a,b]$.


For semelparous populations,  individuals do one of three things: die,  survive and transition to another individual state, or survive and reproduce. The events are mutually exclusive. Let $s_t(x)$ be the probability that an individual in state $x$ survives from time $t$ to time $t+1$. A surviving individual in state $x$ either reproduces with probability $r_t(x)$ or grows with the complementary probability $1-r_t(x)$. For non-reproducing individuals, let $G_t(y,x)$ be the transition or ``growth'' kernel that describes the probability of transitioning from state $x$ to state $y$ for a surviving, non-reproducing individual at time $t$. Roughly $G_t(x,y)dy$ corresponds to the probability of an individual of state $x$ transitioning to a state in the interval $[y,y+dy]$. To describe reproduction, we assume that the distribution of offspring produced by individual depends on their state $x$ and offspring sizes are draw independently from a distribution that only depends on the parents state. For Humboldt Bay wallflower, this is a reasonable assumption as there are not significant differences in seed sizes among individuals (a lll are quite small) and, consequently, it is likely that the size of an offspring in the next census is independent of the sizes of other offspring from the same parent (i.e. sizes are primarily environmentally determined). To describe this reproductive process, let $f_k(x)$ be the probability an individual in state $x$ has $k$ offspring and $K_F(x,y)$ be the kernel that describes the probability of that an offspring of an individual in state $x$ is born into state $y$.  

Let $\s_t=( n_1,\dots, n_k, x_1\dots , x_k)\in \S$ be the population state at time $t$ which updates stochastically as follows 
\begin{enumerate}
\item each individual of type $x_i$ dies with probability $1-p_T(x_i)-p_F(x_i)$.
\item each surviving individual of type $x_i$ reproduces with probability $\frac{p_F(x_i)}{p_F(x_i)+p_T(x_i)}$ or transitions with the complementary probability, 
\item each transitioning individual of type $x_i$  transitions to a state in set $A\subset X$ with probability $\int_A K_T(x_i,y)dy$
\item each reproducing individual of type $x_i$  produces $n$ offspring whose states lie in the sets $A_1,A_2,\dots A_n$ with probability $f_n(x) \int_{A_1} K_F(x,y)dy\dots \int_{A_n} K_F(x,y)dy$
\end{enumerate}


For models of this type, the dynamics of extinction can be described explicitly. These dynamics depend o the probability generating functional for the stochastic process. To describe this probability generating functional, let 
\[
\phi_x(s) = \sum_{k=0}^\infty f_k s^k
\]
be the probability generating function for the offspring number of an individual in state $x$. Then the probability generating functional $\Psi$ for stochastic process $\s_t$ takes continuous functions $h:X\to [0,1]$ to continuous functions from $X$ to $[0,1]$. This functional is defined by (see Appendix)
\[
\Psi (h) (x) =1-p_T(x)-p_F(x)+  p_T(x) \int_X K_T(x,y) h(y) dy + p_F(x)\phi_x\left( \int_X K_F(x,y) h(y)dy\right)
\]
By work of Harris, the probability of being extinct at time $t$ when there is initially one individual of type $x$ is given by 
\[
\Psi^t(\mathbf 0)(x):=\underbrace{\Psi \circ \Psi \circ \dots \circ \Psi}_{t\,\,\rm{ times}}({\mathbf 0})(x)
\]
where $\mathbf 0$ denotes the zero function i.e. $\mathbf 0(x)=0$ for all $x$. In particular, the probability of eventual extinction is given by 
\[
q(x)=\lim_{t\to\infty} \Psi^t (\mathbf 0)(x)
\]
It can be shown that the extinction function  $q:X\to [0,1]$ is the smallest solution to the equation 
\[
\Psi (q) = q
\]
The first characterization of $q$ is useful for numerical implementation. The second characterization is useful for developing sensitivity and elasticity formulas. 

A useful result about $q$ is a threshold theorem due to Harris. Namely, consider the mean-field IPM for this stochastic process:
\[
n_{t+1}(x)=\int M(y,x) n_t(y) dy
\]
where
\[
M(x,y)=  p_T(x) K_T(x,y)  + p_F(x) \phi_x'(1) K_F(x,y) 
\]
If the dominant eigenvalue associated with this IPM is greater than one, then (under suitable technical conditions) the asymptotic probability of extinction is strictly less than one i.e. $q(x)<1$ for all $x$. Alternatively, if the dominant eigenvalue is $\le 1$, then extinction in inevitable i.e. $q(x)=1$ for all $x$. 

\subsection*{Fluctuating environments} These ideas are easily extended to fluctuating environments (see Appendices for details). For the semelparous model, we have the probabilities $p_T(x,t),p_F(x,t)$, the kernels $K_F(x,y,t),K_T(x,y,t)$, and the offspring probability generation function $\phi_x(s,t)$ are now time dependent. These time dependences result in the time-dependent probability generating functional for the stochastic process:
\[
\begin{aligned}
\Psi_t(h)(x)
=&1-p_T(x,t)-p_F(x,t)+  p_T(x,t) \int_X K_T(x,y,t) h(y) dy \\ &+ p_F(x,t)\phi_x\left( \int_X K_F(x,y,t) h(y)dy,t\right)\\
\end{aligned}
\]
The only subtlety in using these time-dependent functionals to determine extinction likelihoods is that one needs to iterate them backwards. Namely, 
\[
\Psi_1 \circ \Psi_2 \circ \dots \Psi_{t-1}\circ \Psi_t({\mathbf 0})(x)=\Psi_1\left(\Psi_2 \left( \dots \Psi_{t-1}\left( \Psi_t({\mathbf 0})\right)\right)\right)(x)
\]
describes the probability of the process going extinct in $t$ time steps given there is initially one individual of type $x$. The probability of eventual extinction is given by 
\[
\lim_{t\to\infty} \Psi_1 \circ \Psi_2 \circ \dots \Psi_{t-1}\circ \Psi_t({\mathbf 0})(x)
\]
Unfortunately, in general, this asymptotic extinction probability can not be expressed implicitly as the solution of a fixed point equation as in the case of constant environments. However, in the case of periodic environments, i.e. $\Psi_{t+k}=\Psi_t$ for all $t$ for some period $k$, the probability of ultimate extinction must satisfy
\[
\Psi_1\circ \Psi_2\circ \dots\circ\Psi_{k-1}\circ \Psi_k (q)(x) = q(x) \mbox{ for all }x
\]
For this periodic case, the Appendix proves there is a threshold theorem. Namely, consider the mean field dynamic
\[
n_{t+1}(x)=\int M_t(y,x) n_t(y) dy
\]
where
\[
M_t(x,y)=  p_T(x,t) K_T(x,y,t)  + p_F(x,t)\frac{\partial  \phi_x}{\partial s}(1,t) K_F(x,y,t) 
\]
If the dominant eigenvalue associated with the time $k$ mapping is greater than one, then $q(x)<1$ for all $x$, else $q(x)=1$ for all $x$. It is natural to conjecture that a similar threshold theorem can be stated in the case of stationary environments. A statement of this conjecture is provided in the Appendices. 



\section*{Numerical implementation}
For this part restrict the discussion to $X=[a,b]$ e.g. continuous size structure. We describe the procedure for approximating a single numerical functional $\Psi$. In the case of fluctuating environments, one needs to approximate each time-dependent functional $\Psi_t$ separately. In the case of the seasonal model, this requires estimating four functionals corresponding to the four seasons. 

To approximate a nonlinear functional $\Psi$ , do the following
\begin{enumerate}
\item Discretize the interval $[a,b]$ into $2n$ subintervals of equal width $\Delta x = \frac{b-a}{2n}$. This means there will $2n+1$ end points $x_i = a+i \Delta x$ for $0\le i \le 2n$. Call the vector of these endpoints xs.
\item For each of the integral operators $h(x)\mapsto \int_X K_\star(x,y) h(y)dy$ where $\star = T$ or $F$ create the matrix approximation using Simpson's rule. Namely create the $2n+1 \times 2n+1$ matrices $\rm{K\star}$ whose $i+1$-th row is 
\[
\frac{\Delta x}{6}
(K_\star(x_i,x_0),4K_\star(x_i,x_1),2K_\star(x_i,x_2),4K_\star(x_i,x_3),\dots, 2K_\star(x_i,x_{2n-2}),4K_\star(x_i,x_{2n-1}),K_\star(x_i,x_{2n}))
\]
If there is no ``eviction'' in the model, the sums of these rows should equal $1$. However, this may not happen. To deal with this issue, define $\star$.evict to be vectors that correspond to row sums of $\rm{K}\star$ and renormalize $\rm{K}\star$ by multiplying it on the left by a diagonal matrix whose diagonal entries are $1/\star.\rm{evict}$. 
\item for each of the $p_\star$ functions create the vectors $\rm{p\star}=p_\star(xs)*\star\rm{.evict}$.
\item create the function $\rm{phi}(x,y)=\phi_x(y)$ which is vector friendly i.e. given vectors x and y it returns a vector of the same length. 
\item create the discretized version of $\Psi$ which takes vectors of length $2n+1$ and returns vectors of length $2n+1$. This function is defined by 
\[
\rm{Psi}(h) = 1 - \rm{pT}-\rm{pF}+\rm{pT}*(AT\o h)+\rm{pF}*\rm{phi}(\rm{xs},AF\o h)
\]
\end{enumerate}

To approximate the extinction probabilities up to time $t$, run a for loop with $q_0$ as a vector of zeros and $q_{t+1}=\rm{Psi}(q_t,t-1)$. For stationary (e.g. constant or periodic) environments, one gets a good approximation of the asymptotic probability of extinction for large enough $t$. 

\section*{Illustration}

The two figures below illustrate output from the wallflower model with disease. Since the eigenvalue for the mean field model is $<1$, eventual extinction occurs with probability one. Figure~\ref{fig:extinction-disease} illustrates probabilities of extinction depend on the season of the founding individual as well as the size of the founding individual. Clearly, initiating a population with larger sized individuals (in general) result in smaller likelihoods of extinction over finite time intervals.  Also initiating populations in Fall or Winter seems to result in lower likelihoods of extinction. Figure~\ref{fig:histogram-disease} predicts that populations generated by the first cohort is most likely to go extinct after 16 years (i.e. 2007) but has a 5\% chance of persisting at least 26 years (i.e. 2017). 

\begin{figure}[h!!!]
\includegraphics[width=\textwidth]{Figs/extinction-disease}
\caption{The probabilities of extinction are plotted as a function of the size of an initial founder. The green curves correspond to the probability of extinction in the first year through twentieth year. The blue curve is the probability of eventual extinction.  }\label{fig:extinction-disease}
\end{figure}

\begin{figure}[h!!!]
\includegraphics[width=\textwidth]{Figs/histogram-disease}
\caption{The probabilities of the time to extinction for the lineage produced by the first cohort of the data set. The solid red curve corresponds to the median extinction time and the dashed red lines correspond to the 5\% and 95\% for extinction times.  }\label{fig:extinction-disease}\label{fig:histogram-disease}
\end{figure}

The next two figures illustrate output from the wallflower model without disease. Since the eigenvalue for the mean field model is $>1$, long-term persistence occurs with positive probability. Figure~\ref{fig:extinction-nodisease} illustrates a similar trend to the disease case, with extinction being now less likely. Figure~\ref{fig:extinction-nodisease2} predicts the necessary population size to ensure long-term persistence. Populations started with seedlings  need on the order of 10,000 individuals to persist, while populations of larger sized individuals only need on the order of 100 to persist. The effect seasonality is most pronounced for populations of seedlings.  
\begin{figure}[h!!!]
\includegraphics[width=\textwidth]{Figs/extinction-nodisease}
\caption{The probabilities of extinction are plotted as a function of the size of an initial founder. The green curves correspond to the probability of extinction in the first year through twentieth year. The blue curve is the probability of eventual extinction.  }\label{fig:extinction-nodisease}
\end{figure}

\begin{figure}[h!!!]
\includegraphics[width=\textwidth]{Figs/extinction-nodisease-2}
\caption{The necessary population size to ensure long-term persistence as a function of the size of individuals and the season}\label{fig:extinction-nodisease2}
\end{figure}




\section*{Appendices}

For iteroparous populations, $p_T(x)$ is the probability of surviving but not reproducing and $p_F(x)$ the probability of surviving and reproducing. Then 
\[
\begin{aligned}
\Psi(h)(x)&=1-p_T(x)-p_F(x)\\
&+\int_X K_T(x,y) h(y) dy \left(p_T(x) + p_F(x)\phi_x\left( \int_X K_F(x,y) h(y)dy\right)\right)\\
\end{aligned}
\]

\end{document}
  